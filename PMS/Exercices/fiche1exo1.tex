\PassOptionsToPackage{unicode=true}{hyperref} % options for packages loaded elsewhere
\PassOptionsToPackage{hyphens}{url}
%
\documentclass[]{article}
\usepackage{lmodern}
\usepackage{amssymb,amsmath}
\usepackage{ifxetex,ifluatex}
\usepackage{fixltx2e} % provides \textsubscript
\ifnum 0\ifxetex 1\fi\ifluatex 1\fi=0 % if pdftex
  \usepackage[T1]{fontenc}
  \usepackage[utf8]{inputenc}
  \usepackage{textcomp} % provides euro and other symbols
\else % if luatex or xelatex
  \usepackage{unicode-math}
  \defaultfontfeatures{Ligatures=TeX,Scale=MatchLowercase}
\fi
% use upquote if available, for straight quotes in verbatim environments
\IfFileExists{upquote.sty}{\usepackage{upquote}}{}
% use microtype if available
\IfFileExists{microtype.sty}{%
\usepackage[]{microtype}
\UseMicrotypeSet[protrusion]{basicmath} % disable protrusion for tt fonts
}{}
\IfFileExists{parskip.sty}{%
\usepackage{parskip}
}{% else
\setlength{\parindent}{0pt}
\setlength{\parskip}{6pt plus 2pt minus 1pt}
}
\usepackage{hyperref}
\hypersetup{
            pdfborder={0 0 0},
            breaklinks=true}
\urlstyle{same}  % don't use monospace font for urls
\usepackage[margin=1in]{geometry}
\usepackage{color}
\usepackage{fancyvrb}
\newcommand{\VerbBar}{|}
\newcommand{\VERB}{\Verb[commandchars=\\\{\}]}
\DefineVerbatimEnvironment{Highlighting}{Verbatim}{commandchars=\\\{\}}
% Add ',fontsize=\small' for more characters per line
\usepackage{framed}
\definecolor{shadecolor}{RGB}{248,248,248}
\newenvironment{Shaded}{\begin{snugshade}}{\end{snugshade}}
\newcommand{\AlertTok}[1]{\textcolor[rgb]{0.94,0.16,0.16}{#1}}
\newcommand{\AnnotationTok}[1]{\textcolor[rgb]{0.56,0.35,0.01}{\textbf{\textit{#1}}}}
\newcommand{\AttributeTok}[1]{\textcolor[rgb]{0.77,0.63,0.00}{#1}}
\newcommand{\BaseNTok}[1]{\textcolor[rgb]{0.00,0.00,0.81}{#1}}
\newcommand{\BuiltInTok}[1]{#1}
\newcommand{\CharTok}[1]{\textcolor[rgb]{0.31,0.60,0.02}{#1}}
\newcommand{\CommentTok}[1]{\textcolor[rgb]{0.56,0.35,0.01}{\textit{#1}}}
\newcommand{\CommentVarTok}[1]{\textcolor[rgb]{0.56,0.35,0.01}{\textbf{\textit{#1}}}}
\newcommand{\ConstantTok}[1]{\textcolor[rgb]{0.00,0.00,0.00}{#1}}
\newcommand{\ControlFlowTok}[1]{\textcolor[rgb]{0.13,0.29,0.53}{\textbf{#1}}}
\newcommand{\DataTypeTok}[1]{\textcolor[rgb]{0.13,0.29,0.53}{#1}}
\newcommand{\DecValTok}[1]{\textcolor[rgb]{0.00,0.00,0.81}{#1}}
\newcommand{\DocumentationTok}[1]{\textcolor[rgb]{0.56,0.35,0.01}{\textbf{\textit{#1}}}}
\newcommand{\ErrorTok}[1]{\textcolor[rgb]{0.64,0.00,0.00}{\textbf{#1}}}
\newcommand{\ExtensionTok}[1]{#1}
\newcommand{\FloatTok}[1]{\textcolor[rgb]{0.00,0.00,0.81}{#1}}
\newcommand{\FunctionTok}[1]{\textcolor[rgb]{0.00,0.00,0.00}{#1}}
\newcommand{\ImportTok}[1]{#1}
\newcommand{\InformationTok}[1]{\textcolor[rgb]{0.56,0.35,0.01}{\textbf{\textit{#1}}}}
\newcommand{\KeywordTok}[1]{\textcolor[rgb]{0.13,0.29,0.53}{\textbf{#1}}}
\newcommand{\NormalTok}[1]{#1}
\newcommand{\OperatorTok}[1]{\textcolor[rgb]{0.81,0.36,0.00}{\textbf{#1}}}
\newcommand{\OtherTok}[1]{\textcolor[rgb]{0.56,0.35,0.01}{#1}}
\newcommand{\PreprocessorTok}[1]{\textcolor[rgb]{0.56,0.35,0.01}{\textit{#1}}}
\newcommand{\RegionMarkerTok}[1]{#1}
\newcommand{\SpecialCharTok}[1]{\textcolor[rgb]{0.00,0.00,0.00}{#1}}
\newcommand{\SpecialStringTok}[1]{\textcolor[rgb]{0.31,0.60,0.02}{#1}}
\newcommand{\StringTok}[1]{\textcolor[rgb]{0.31,0.60,0.02}{#1}}
\newcommand{\VariableTok}[1]{\textcolor[rgb]{0.00,0.00,0.00}{#1}}
\newcommand{\VerbatimStringTok}[1]{\textcolor[rgb]{0.31,0.60,0.02}{#1}}
\newcommand{\WarningTok}[1]{\textcolor[rgb]{0.56,0.35,0.01}{\textbf{\textit{#1}}}}
\usepackage{graphicx,grffile}
\makeatletter
\def\maxwidth{\ifdim\Gin@nat@width>\linewidth\linewidth\else\Gin@nat@width\fi}
\def\maxheight{\ifdim\Gin@nat@height>\textheight\textheight\else\Gin@nat@height\fi}
\makeatother
% Scale images if necessary, so that they will not overflow the page
% margins by default, and it is still possible to overwrite the defaults
% using explicit options in \includegraphics[width, height, ...]{}
\setkeys{Gin}{width=\maxwidth,height=\maxheight,keepaspectratio}
\setlength{\emergencystretch}{3em}  % prevent overfull lines
\providecommand{\tightlist}{%
  \setlength{\itemsep}{0pt}\setlength{\parskip}{0pt}}
\setcounter{secnumdepth}{0}
% Redefines (sub)paragraphs to behave more like sections
\ifx\paragraph\undefined\else
\let\oldparagraph\paragraph
\renewcommand{\paragraph}[1]{\oldparagraph{#1}\mbox{}}
\fi
\ifx\subparagraph\undefined\else
\let\oldsubparagraph\subparagraph
\renewcommand{\subparagraph}[1]{\oldsubparagraph{#1}\mbox{}}
\fi

% set default figure placement to htbp
\makeatletter
\def\fps@figure{htbp}
\makeatother


\author{}
\date{\vspace{-2.5em}}

\begin{document}

\hypertarget{principes-et-mthodes-statistiques}{%
\section{Principes et M?thodes
Statistiques}\label{principes-et-mthodes-statistiques}}

\hypertarget{fiche-1---exercice-1}{%
\section{Fiche 1 - Exercice 1}\label{fiche-1---exercice-1}}

Soit \(X\) une variable al?atoire de loi normale
\({\cal N}(m,\sigma^2)\), \(m \in R, \sigma \in R^{+*}\). \(X\) est ?
valeurs dans \(R\), son esp?rance est \(E[X]=m\) et sa variance est
\(Var[X]=\sigma^2\).

La densit? de la loi \({\cal N}(m,\sigma^2)\) est :
\[f_X(x) = {\displaystyle \frac{1}{\sigma \sqrt{2\pi}}} \, e^{  - \frac{(x - m)^2}{2\sigma^2}}\]

Sa fonction de r?partition n'a pas d'expression explicite :
\[F_X(x) = \int_{-\infty}^x f_X(u) du = \int_{-\infty}^x {\displaystyle \frac{1}{\sigma \sqrt{2\pi}}} \, e^{- \frac{(u - m)^2}{2\sigma^2}} du\]

\hypertarget{question-1}{%
\subsection{Question 1}\label{question-1}}

Tracer les densit?s des lois \({\cal N}(2,1)\), \({\cal N}(-3,1)\) et
\({\cal N}(-3,9)\).

\begin{Shaded}
\begin{Highlighting}[]
\NormalTok{m<-}\DecValTok{2}
\NormalTok{sigma<-}\DecValTok{1}
\KeywordTok{curve}\NormalTok{(}\KeywordTok{dnorm}\NormalTok{(x,m,sigma),}\OperatorTok{-}\DecValTok{12}\NormalTok{,}\DecValTok{12}\NormalTok{, }\DataTypeTok{col=}\StringTok{"blue"}\NormalTok{)}

\NormalTok{m<-}\OperatorTok{-}\DecValTok{3}
\KeywordTok{curve}\NormalTok{(}\KeywordTok{dnorm}\NormalTok{(x,m,sigma),}\DataTypeTok{add=}\NormalTok{T, }\DataTypeTok{col=}\StringTok{"red"}\NormalTok{)}

\NormalTok{sigma<-}\DecValTok{3}
\KeywordTok{curve}\NormalTok{(}\KeywordTok{dnorm}\NormalTok{(x,m,sigma),}\DataTypeTok{add=}\NormalTok{T, }\DataTypeTok{col=}\StringTok{"green"}\NormalTok{)}
\end{Highlighting}
\end{Shaded}

\includegraphics{fiche1exo1_files/figure-latex/unnamed-chunk-1-1.pdf}

On remarque que l'aire sous la densit? est constante (elle vaut 1).

\hypertarget{question-2}{%
\subsection{Question 2}\label{question-2}}

Tracer les fonctions de r?partition de ces m?mes lois. Calculer
\(P(X \leq 0)\) dans les 3 cas.

\begin{Shaded}
\begin{Highlighting}[]
\NormalTok{m<-}\DecValTok{2}
\NormalTok{sigma<-}\DecValTok{1}
\KeywordTok{curve}\NormalTok{(}\KeywordTok{pnorm}\NormalTok{(x,m,sigma),}\OperatorTok{-}\DecValTok{12}\NormalTok{,}\DecValTok{12}\NormalTok{, }\DataTypeTok{col=}\StringTok{"blue"}\NormalTok{)}
\KeywordTok{pnorm}\NormalTok{(}\DecValTok{0}\NormalTok{,m,sigma)}
\end{Highlighting}
\end{Shaded}

\begin{verbatim}
## [1] 0.02275013
\end{verbatim}

\begin{Shaded}
\begin{Highlighting}[]
\NormalTok{m<-}\OperatorTok{-}\DecValTok{3}
\KeywordTok{curve}\NormalTok{(}\KeywordTok{pnorm}\NormalTok{(x,m,sigma),}\DataTypeTok{add=}\NormalTok{T, }\DataTypeTok{col=}\StringTok{"red"}\NormalTok{)}
\KeywordTok{pnorm}\NormalTok{(}\DecValTok{0}\NormalTok{,m,sigma)}
\end{Highlighting}
\end{Shaded}

\begin{verbatim}
## [1] 0.9986501
\end{verbatim}

\begin{Shaded}
\begin{Highlighting}[]
\NormalTok{sigma<-}\DecValTok{3}
\KeywordTok{curve}\NormalTok{(}\KeywordTok{pnorm}\NormalTok{(x,m,sigma),}\DataTypeTok{add=}\NormalTok{T, }\DataTypeTok{col=}\StringTok{"green"}\NormalTok{)}
\end{Highlighting}
\end{Shaded}

\includegraphics{fiche1exo1_files/figure-latex/unnamed-chunk-2-1.pdf}

\begin{Shaded}
\begin{Highlighting}[]
\KeywordTok{pnorm}\NormalTok{(}\DecValTok{0}\NormalTok{,m,sigma)}
\end{Highlighting}
\end{Shaded}

\begin{verbatim}
## [1] 0.8413447
\end{verbatim}

\hypertarget{question-3}{%
\subsection{Question 3}\label{question-3}}

Soit \(U\) une variable al?atoire de loi normale centr?e-r?duite
\({\cal N}(0,1)\). Tracer la densit? de \(U\). On note \(\phi\) sa
fonction de r?partition.

Calculer \(P(U \leq 0)\), \(P(U \leq 1.25)\), \(P(U \leq 1)\) et
\(P(U \leq -1)\). Retrouver ces r?sultats ? l'aide de la table 1 de la
loi normale centr?e-r?duite en 8.2.1.

\begin{Shaded}
\begin{Highlighting}[]
\NormalTok{m<-}\DecValTok{0}
\NormalTok{sigma<-}\DecValTok{1}
\KeywordTok{curve}\NormalTok{(}\KeywordTok{dnorm}\NormalTok{(x,m,sigma),}\OperatorTok{-}\DecValTok{4}\NormalTok{,}\DecValTok{4}\NormalTok{)}
\KeywordTok{abline}\NormalTok{(}\DataTypeTok{h=}\DecValTok{0}\NormalTok{)}

\KeywordTok{pnorm}\NormalTok{(}\DecValTok{0}\NormalTok{)}
\end{Highlighting}
\end{Shaded}

\begin{verbatim}
## [1] 0.5
\end{verbatim}

\begin{Shaded}
\begin{Highlighting}[]
\KeywordTok{lines}\NormalTok{(}\KeywordTok{c}\NormalTok{(}\DecValTok{0}\NormalTok{,}\DecValTok{0}\NormalTok{),}\KeywordTok{c}\NormalTok{(}\DecValTok{0}\NormalTok{,}\KeywordTok{dnorm}\NormalTok{(}\DecValTok{0}\NormalTok{)),}\DataTypeTok{col=}\StringTok{"red"}\NormalTok{,}\DataTypeTok{lwd=}\DecValTok{2}\NormalTok{)}

\KeywordTok{pnorm}\NormalTok{(}\FloatTok{1.25}\NormalTok{)}
\end{Highlighting}
\end{Shaded}

\begin{verbatim}
## [1] 0.8943502
\end{verbatim}

\begin{Shaded}
\begin{Highlighting}[]
\KeywordTok{lines}\NormalTok{(}\KeywordTok{c}\NormalTok{(}\FloatTok{1.25}\NormalTok{,}\FloatTok{1.25}\NormalTok{),}\KeywordTok{c}\NormalTok{(}\DecValTok{0}\NormalTok{,}\KeywordTok{dnorm}\NormalTok{(}\FloatTok{1.25}\NormalTok{)),}\DataTypeTok{col=}\StringTok{"blue"}\NormalTok{,}\DataTypeTok{lwd=}\DecValTok{2}\NormalTok{)}
\end{Highlighting}
\end{Shaded}

\includegraphics{fiche1exo1_files/figure-latex/unnamed-chunk-3-1.pdf}

\begin{Shaded}
\begin{Highlighting}[]
\KeywordTok{pnorm}\NormalTok{(}\DecValTok{1}\NormalTok{)}
\end{Highlighting}
\end{Shaded}

\begin{verbatim}
## [1] 0.8413447
\end{verbatim}

\begin{Shaded}
\begin{Highlighting}[]
\DecValTok{1}\OperatorTok{-}\KeywordTok{pnorm}\NormalTok{(}\DecValTok{1}\NormalTok{)}
\end{Highlighting}
\end{Shaded}

\begin{verbatim}
## [1] 0.1586553
\end{verbatim}

\begin{Shaded}
\begin{Highlighting}[]
\KeywordTok{pnorm}\NormalTok{(}\OperatorTok{-}\DecValTok{1}\NormalTok{)}
\end{Highlighting}
\end{Shaded}

\begin{verbatim}
## [1] 0.1586553
\end{verbatim}

\hypertarget{question-4}{%
\subsection{Question 4}\label{question-4}}

Calculer \(u_{0.05} = \phi^{-1}(1-0.05/2)\) et
\(u_{0.25} = \phi^{-1}(1-0.25/2)\). Retrouver ces r?sultats ? l'aide de
la la table 2 de la loi normale centr?e-r?duite en 8.2.2.

\begin{Shaded}
\begin{Highlighting}[]
\KeywordTok{qnorm}\NormalTok{(}\DecValTok{1}\FloatTok{-0.05}\OperatorTok{/}\DecValTok{2}\NormalTok{)}
\end{Highlighting}
\end{Shaded}

\begin{verbatim}
## [1] 1.959964
\end{verbatim}

\begin{Shaded}
\begin{Highlighting}[]
\KeywordTok{qnorm}\NormalTok{(}\DecValTok{1}\FloatTok{-0.25}\OperatorTok{/}\DecValTok{2}\NormalTok{)}
\end{Highlighting}
\end{Shaded}

\begin{verbatim}
## [1] 1.150349
\end{verbatim}

\begin{Shaded}
\begin{Highlighting}[]
\KeywordTok{curve}\NormalTok{(}\KeywordTok{dnorm}\NormalTok{(x,m,sigma),}\OperatorTok{-}\DecValTok{4}\NormalTok{,}\DecValTok{4}\NormalTok{)}
\KeywordTok{abline}\NormalTok{(}\DataTypeTok{h=}\DecValTok{0}\NormalTok{)}
\KeywordTok{lines}\NormalTok{(}\KeywordTok{c}\NormalTok{(}\KeywordTok{qnorm}\NormalTok{(}\DecValTok{1}\FloatTok{-0.05}\OperatorTok{/}\DecValTok{2}\NormalTok{),}\KeywordTok{qnorm}\NormalTok{(}\DecValTok{1}\FloatTok{-0.05}\OperatorTok{/}\DecValTok{2}\NormalTok{)),}\KeywordTok{c}\NormalTok{(}\DecValTok{0}\NormalTok{,}\KeywordTok{dnorm}\NormalTok{(}\KeywordTok{qnorm}\NormalTok{(}\DecValTok{1}\FloatTok{-0.05}\OperatorTok{/}\DecValTok{2}\NormalTok{))),}\DataTypeTok{col=}\StringTok{"blue"}\NormalTok{,}\DataTypeTok{lwd=}\DecValTok{2}\NormalTok{)}
\KeywordTok{lines}\NormalTok{(}\KeywordTok{c}\NormalTok{(}\OperatorTok{-}\KeywordTok{qnorm}\NormalTok{(}\DecValTok{1}\FloatTok{-0.05}\OperatorTok{/}\DecValTok{2}\NormalTok{),}\OperatorTok{-}\KeywordTok{qnorm}\NormalTok{(}\DecValTok{1}\FloatTok{-0.05}\OperatorTok{/}\DecValTok{2}\NormalTok{)),}\KeywordTok{c}\NormalTok{(}\DecValTok{0}\NormalTok{,}\KeywordTok{dnorm}\NormalTok{(}\KeywordTok{qnorm}\NormalTok{(}\DecValTok{1}\FloatTok{-0.05}\OperatorTok{/}\DecValTok{2}\NormalTok{))),}\DataTypeTok{col=}\StringTok{"blue"}\NormalTok{,}\DataTypeTok{lwd=}\DecValTok{2}\NormalTok{)}
\end{Highlighting}
\end{Shaded}

\includegraphics{fiche1exo1_files/figure-latex/unnamed-chunk-4-1.pdf}

\end{document}
