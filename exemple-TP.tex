\documentclass[12pt]{book}
% charge un formatage standard pour le document contenu dans le fichier
% article.cls
% à la place de "article" on peut aussi mettre report, book,...

% chargement des packages
\usepackage[utf8]{inputenc} % permet d'écrire des caractère spéciaux dans le *.tex
\usepackage[T1]{fontenc} % encodage de ces caractères spéciaux (ç,é,ï,...)
\usepackage[french]{babel} % formatage typique à une langue
\usepackage{amsmath,amssymb,amsthm} % packages utiles pour les mathématiques
\usepackage{listings}
%\lstset{language=Python} % pour inclure du code python...
\lstset{language=TeX} % pour inclure du code LaTeX
\usepackage{graphicx} % pour inclure des images
\textheight20cm % pour régler la hauteur utile du texte

% définir des macros [un plus par rapport aux notebooks]
\def\bbR{{\mathbb{C}}} % ensemble des complexes, \mathbb nécessite le package amssymb
\def\bbC{{\mathbb{R}}} % ensemble des réels
\newtheorem{theorem}{Théorème} % on définit un environnement de théorème

% informations de titre
\title{Un exemple de fichier LaTeX}
\author{Brigitte Bidegaray-Fesquet, Guillaume James, Adam Larat, \\
Hubert Leterme, Valérie Perrier et Simon Santoso}
\date{\today} % la date du jour, on peut mettre n'importe quelle autre date

% A partir d'ici commence vraiment le document...
\begin{document}
\maketitle % affiche les informations de titre

\section{Apprenons par l'exemple, un extrait du TP}
% commence une section la numérotation se fait automatiquement

\subsection{Les séries de Fourier}
% sous-section, numérotée automatiquement aussi
Soit $f:\bbR\mapsto\bbC$ une fonction $T$ périodique, la série de Fourier de $f$ s'écrit comme une combinaison linéaire de fonctions sinusoïdales:
% l'espace avant les deux points correspondant à la norme française
% sont automatiquement ajoutés grâce à \usepackage[french]{babel}
\begin{equation}
	f(t) = \sum_{n=-\infty}^\infty c_n(f) \exp\left(2i\pi\frac{n}{T}t\right),
\end{equation}
% contrairement à dans le notebook, cette équation sera numérotée, pour ne pas % numéroter utiliser {equation*} disponible grâce à \usepackage{amsmath}
où les coefficients $c_n(f)$ de $f$ sont appelés \textit{coefficients de Fourier}
et sont définis comme suit:
\begin{equation}
	c_n(f) = \frac1T \int_{-T/2}^{T/2} f(t) \exp\left(-2i\pi\frac{n}{T}t\right) dt.
\end{equation}
Dans le cas de fonctions $f:\bbR\mapsto\bbR$, quelques simplifications
peuvent être faites, la série de Fourier peut s'écrire comme suit:
\begin{equation}
	f(t) = a_0(f)
    + \sum_{n=1}^\infty \left[
      a_n(f) \cos\left(2\pi\frac{n}{T}t\right)
    + b_n(f) \sin\left(2\pi\frac{n}{T}t\right)\right],
\end{equation}
où les coefficients $a_n(f)$ et $b_n(f)$, sont définis comme suit:
\begin{equation}
\label{eq:coeffs_a_et_b}
	\left\{
    \begin{aligned}
		a_0(f) & = \frac1T \int_{-T/2}^{T/2} f(t) dt, \\
		a_n(f) & = \frac2T \int_{-T/2}^{T/2} f(t) \cos\left(2\pi\frac{n}{T}t\right) dt, \\
		b_n(f) & = \frac2T \int_{-T/2}^{T/2} f(t) \sin\left(2\pi\frac{n}{T}t\right) dt.
	\end{aligned}
    \right.
\end{equation}

\subsection{Exemple de théorème}

\begin{theorem}
\label{th:Fourier}
La fonction 1-périodique $f$, définie sur $[-\frac12, \frac12[$ par
\begin{equation*}
    f(x) = \begin{cases}
    -1 & \text{pour } -\frac12 \leq x \leq 0, \\
    1 & \text{pour } 0 < x < \frac12
    \end{cases}
\end{equation*}
admet pour série de Fourier
\begin{equation*}
    f(x) = \frac4\pi \sum_{n=0}^{+\infty} \frac{\sin(2(2n+1)\pi x)}{2n+1}.
\end{equation*}
\end{theorem}

\begin{proof}
Utiliser la formule \eqref{eq:coeffs_a_et_b}.
\end{proof}

\section{Présenter ses résultats}

\subsection{Les images}

La figure \ref{fig:exemple} permet d'illustrer le théorème \ref{th:Fourier}.
Comme vous le voyez la figure est flottante~: elle n'apparaît pas forcément là où elle est déclarée.
Les flottants sont numérotés automatiquement.

\begin{figure}[t]
\centering
\includegraphics[width=.5\textwidth]{sampleimage.pdf}
\caption{\label{fig:exemple}Une figure d'exemple.}
\end{figure}

\subsection{Les tableaux}

Les tableaux sont une autre sorte de flottants. Un autre exemple issu du TP est donné  dans la table \ref{tab:operations}.

\begin{table}[t]
\centering
\begin{tabular}{l|l}
\textbf{commande} & \textbf{description} \\
\hline
\verb!A[k,:]! & $k$-ième ligne de la matrice $A$ \\
\verb!A + B! & somme \\
\verb!np.dot(A,B)! & produit \\
\end{tabular}
\caption{\label{tab:operations}Quelques opérations sur les matrices.}
\end{table}

\subsection{Les codes}

La table \ref{tab:operations} donne un tout petit exemple de représentation brut de code.
Pour les plus gros exemples il y a l'environnement \verb!verbatim!, mais on peut encore faire mieux avec le package \verb!listings!.
Par exemple le code pour inclure une figure est le suivant:

\begin{lstlisting}
\begin{figure}[t]
\centering
\includegraphics[width=.5\textwidth]{sampleimage.pdf}
\caption{\label{fig:exemple}Une figure d'exemple.}
\end{figure}
\end{lstlisting}

\section{Pour finir...}

Un rapport en bonne et due forme doit citer ses références. Clairement, nous avons utilisé ici \cite{ref1}.

\bibliographystyle{plain} % correspond au style plain.bst,... il y en a d'autres bien sûr
\begin{thebibliography}{9}
\bibitem{ref1} Brigitte Bidegaray-Fesquet, Guillaume James, Adam Larat, Hubert Leterme, Valérie Perrier, Simon Santoso. \textit{Mini-stage d'analyse pour l'ingénieur.}
Notebook Python. Consulté le 28 octobre 2019.
\end{thebibliography}

\end{document}
